% Szkielet pracy:
%
% Copyright (c) 2001 by Marcin Woliñski <M.Wolinski@gust.org.pl>
% Poprawki spowodowane zmianami przepisów - Marcin Szczuka, 1.10.2004
% Poprawki spowodowane zmianami przepisow i ujednolicenie 
% - Seweryn Kar³owicz, 05.05.2006
% dodaj opcjê [licencjacka] dla pracy licencjackiej
%

\documentclass[licencjacka]{pracamgr}

\usepackage{polski}
\usepackage[utf8]{inputenc}
\usepackage{verbatim}

% Tutaj każdy wstawia swoje dane drukując pracę licencjacką
\author{Jakub Oćwieja}
\nralbumu{292690}

\title{Pengurus - system zarządzania biurem tłumaczeń}

\tytulang{Pengurus - translation agency management system}

\kierunek{Informatyka}

% Praca wykonana pod kierunkiem:
% (poda¿ tytu¿/stopie¿ imi¿ i nazwisko opiekuna
% Instytut
% ew. Wydzia¿ ew. Uczelnia (je¿eli nie MIM UW))
\opiekun{Michał Możdżonek}

\date{Maj 2012}

%Dziedzina wg klasyfikacji Socrates-Erasmus:
\dziedzina{ 
11.3 Informatyka\\ 
}

%Klasyfikacja tematyczna wedlug AMS (matematyka) lub ACM (informatyka)
\klasyfikacja{D. Software\\
  D.2. Software engineering\\
  D.2.9 Management}

% S¿owa kluczowe:
\keywords{system, zarządzanie, tłumacz, tłumaczenie, aplikacja, firma, biuro}

% Miejsce na w¿asne makra i ¿rodowiska:
\newtheorem{defi}{Definicja}[section]

% koniec definicji

\begin{document}
\maketitle

%tu idzie streszczenie na strone poczatkowa
\begin{abstract}
  W pracy przedstawiono przebieg projektowania i implementacji projektu Pengurus, czyli systemu do zarządzania biurem tłumaczeń.
  W kolejnych rozdziałach opisane zostały nasz projekt, odniesienie do konkurencyjnych produktów, wyniki pracy, użyte narzędzia i technologie oraz sam przebieg wykonania projektu.
\end{abstract}

\tableofcontents
%\listoffigures
%\listoftables

\chapter{Wprowadzenie - Wizja}
\section{Miejsce na rynku}
W dzisiejszym świecie bardzo istotną kwestią jest usprawnianie zarządzania firmą. 
Najważniejsze jest przyspieszenie i sformalizowanie przebiegu procesu biznesowego. 
Najłatwiejszym i najtańszym sposobem jest wykorzystanie nowych technologii i zinformatyzowanie procesu. \\


Z takim problemem muszą zmierzyć się również biura tłumaczeń. Duża konkurencja wymaga od nich usprawnienie procesu tłumaczenia, 
łatwy kontakt z klientem oraz pracownikami nietatowymi, którzy zazwyczaj pracują poza siedzibą firmy , 
możliwość wprowadzenia szybkich korekt. Ponadto wymagane jest bezpieczeństwo przechowywania ( danych osobowych, plików, własności intelektualnej ),
prostota użytkowania i powszechny dostęp. \\


Na rynku znajdują się projekty, które są przeznaczone dla biur tłumaczeń.
Są to jednak produkty, które są ograniczone jedynie do pracowników biura ( nie jest możliwa praca mobilna ), 
bądź w przypadku projektów webowych są napisane w starych technologiach (PHP) , zapewniają małe bezpieczeństwo i nie są przeznaczone do kontaktów z klientami.
Pengurus CRM ma być aplikacją, która rozwiązuje te problemy. 
Będzie to pierwsza na rynku aplikacja WWW przeznaczona bezpośrednio dla biur tłumaczeń, w której wykorzystano nowe technnologie (GWT, Hibernate, Spring),
przeznaczona do kontaktów z pracownikami oraz klientami. Do obsługi będzie wystarczać przeglądarka WWW zatem umożliwi to  mobilną pracę 
oraz skróci klientowi czas jaki musi poświęcić na nadzorowanie projektu. 

\section{Użytkownicy}
\begin{itemize}
\item Kierownik (Executive) – pracownik firmy, posiadający najwięcej praw w systemie Pengurus, fizycznie rola przewidziana dla osób postawionych najwyżej w hierarchii firmy. Jako jedyny posiada dostęp do panelu administratora.
\item Menadżer Projektu (Project Manager)– pracownik firmy, odpowiadający za nadzorowanie projektów.
\item Ekspert (Expert)– pracownik firmy, odpowiedzialny za tłumaczenie dokumentów, zleconych przez Menadżera Projektu (Tłumacz).
\item Księgowy (Accountant)– pracownik firmy, odpowiedzialny za księgowość firmy.
\item Pracownik nieetatowy (Freelancer)– osoba nie będąca na stałe zatrudniona w firmie, która na potrzeby chwili wykonuje odpłatnie zlecone tłumaczenia (Tłumacz).
\item Klient (Client)– klient indywidualny bądź biznesowy zlecający tłumaczenie firmie.
\end{itemize}

\section{Przypadki użycia}
\begin{itemize} 
\item Przyjęcie zlecenia

Kierownik zlecenia ( Menadżer ds. produkcji/ Stanowisko ds. relacji z klientami i produkcji ) 
jest odpowiedzialny za przyjęcie zlecenia od klienta oraz przygotowanie odpowiedniej oferty, 
która następnie zostaje podana ocenie klientowi.


Kierownik zlecenia odbiera dokumenty od klienta wraz z tekstem oryginału , zapoznaje się on z przesłanym/odebranym materiałem,
a następnie ustala ze zlecającym szczegóły dotyczące zlecenia takie jak data realizacji, cena,  tryb zlecenia, rodzajem tłumaczenia,  
materiały pomocnicze.



Kierownik zlecenia przgotowuje dziennik zlecenia. W dzienniku zapisywana jest historia pracy z dokumentem i 
odnotowywane są wszystkie istotne zdarzenia mające wpływ na jakość i/lub bezpieczeństwo informacji przekazanych przez klienta. 
Ponadto przechowywane są : data utworzenia zlecenia, data przyjęcia tekstu źródłowego oraz związanych z nim materiałów,
,numer zlecenia, informację o podjętych działaniach w celu usunięcia niezgodności. 

Przed rozpoczęciem realizacji projektu klient jest zobowiązany do zaakceptowania przygotowanej oferty.

\item Rejestracja zlecenia \\
Kierownik zlecenia wyznacza lidera/liderów dla tłumaczenia (Menadżera Projektu). W uzasadnionych przypadkach działania Lidera wykonuje kierownik zlecenia.
Ponadto jest on odpowiedzialny za przypisanie odpowiednich tłumaczy/ weryfikatorów do projektu
Kierownik ponadto udostępnia menadżerom przesłane przez klienta zasoby ( pliki ) i wyznacza im temin realizacji poszczygólnych etapów projektu.

\item Tłumaczenie i weryfikacja\\
Menadżer Projektu przekazuje odpowiednim tłumaczom materiały oraz wyznacza termin realizacji tłumaczenia.
W tym celu tworzy nowe pliki i udostępnia je tłumaczowi.
Ponadto menadżer wyznacza odpwiedniego Weryfikatora.
Tłumacz przesyła pliki ze zrealizownym etapami pracy. Każdy z tych plików może być edytowany przez autora, a dostęp do nich posiadają menadżerowie oraz weryfikatorzy.
Po zakończonym tłumaczeniu weryfikator sprawdza wykonaną pracę, a następnie wysyła plik z poprawkami zapisuje uwagi oraz akceptuje wykonaną pracę lub nakazuje ponowne rozpatrzenie.
Po ostatecznym zaakceptowaniu przez weryfikator menadżer odbiera pliki.
W przypadku pracowników nieetatowych następuje rozliczenie z tłumaczem, które jest potwierdzane przez księgowego.

\item Przekazanie gotowego tłumaczenia do zleceniodawcy\\
Menadżer projektu odbiera od tłumaczy i weryfikatorów pliki oraz przygotowuje ostateczny dokument, który następnie zostaje podany ocenie kierownikowi zlecenia.
Kieronik zlecenia po zaakceptowaniu przekazuje go klientowi, który podaje go swojej ocenie. W razie reklamacji dokument wraz z uwagami zostaje przesłany 
menadżerowi projektu do ponownego rozpatrzenia.

\item Archiwizacja zleceń\\
Po zamknięciu obsługi zlecenia – formularz zlecenia przechowywany jest,
wraz z adnotacją dotyczącą danych faktury oraz danych określających uczestników 
procesu realizacji, w wersji papierowej w kolejności odpowiadającej numerom 
wewnętrznym nadanym na początku procesu.
Pliki danego zlecenia wraz z dziennikiem zlecenia, przechowywane są w 
katalogu przeznaczonym do obsługi danego zlecenia na serwerze.
Dane znajdujące się na serwerze poddawane są archiwizacji, zgodnie z 
Polityką tworzenia kopii zapasowych Pl-4.1

\item Tworzenie i edycja użytkowników systemu\\
Kierownik tworzy użytkownika oraz nadaje mu odpowiednie role ( tłumacz , klient, menadżer projektu, itd. ).
Pracownik posiada : login, dane osobowe i kontaktowe , posiadane umiejętności tłumaczenia ( razem z posiadanymi prawamy - tłumaczenia sądowe, techniczne ).
Klient może być kilentem biznesowym ( posiada pośredników ) , bądź klientem indywidualnym.

\end{itemize}



\chapter{Aplikacja}
\section{Definicje}
Tutaj zdefiniujemy rzeczy, o których będziemy mówić na kolejnych stronach.
\section{Opis aplikacji}
Opis tego jak ostatecznie wygląda nasza aplikacja. Nie wiem jak bardzo szczegółowo - to w zależności od tego jakie szczegóły będziemy chcieli opisać w dalszej części pracy. Chociaż myślę, że możemy tu napisać i oscreenować wszystko, co robi aplikacja (ale tylko zewnętrzne cechy). Chyba, że ten topic przeniesiemy w późniejsze miejsce. Ale jak dla mnie tu jest idealnie.
\section{Podstawowy przypadek użycia}

\chapter{Rozwiązania}
\section{Spring Framework}

Spring Framework jest jedną z najpopularniejszych i najlepiej działających platform programistycznych dla Javy. Zalet tego frameworku jest wiele, poniżej przedstawione zostały jednak te najbardziej zauważalne w pracy nad naszym projektem.

\paragraph{Wzorzec projektowy MVC}
Spring MVC jest jednym z pakietów oferowanych przez Spring Framework. Został on zaprojektowany do tworzenia aplikacji działających na bazie wzorca projektowego Model-View-Controller i oferuje wiele mechanizmów ułatwiających pracę z nimi. Przebieg akcji z użyciem Spring MVC w naszej aplikacji wygląda następująco:

\begin{itemize}
  \item Klient prosi o zasoby aplikację
  \item Zapytanie jest przechwytywane przez Spring Front Controller które zajmuje się znalezieniem odpowiednich mapowań adresów
  \item Mapowania wykorzystywane są do znalezienia odpowiedniego kontrolera aplikacji odpowiadającego za odpowiedzenia na zapytanie
  \item Kontroler przetwarza zapytanie i generuje lub odnajduje dane, o które poproszono w zapytaniu
  \item Stworzony widok zostaje odesłany do klienta
\end{itemize}

Dodatkową zaletą implementacji tego modelu przez Spring jest fakt, że pozwala on na integrację z innymi frameworkami i wybór obiektów odpowiedzialnych za wykonanie poszczególnych akcji, co jest jedną z większych motywacji. W projekcie bowiem wykorzystujemy to do połączenia tego modelu z możliwościami oferowanymi przez frameworki Hibernate i ExtGWT.

To właśnie dlatego w przepływie tym pominięte zostało generowanie widoku na podstawie danych, gdyż klientem jest samodzielny framework prezentujący dane użytkownikowi.

\paragraph{Inwersja kontroli}
Inwersja kontroli (ang. IoC) to w programowaniu obiektowym praktyka, w której powiązania obiektów ustalane są dopiero podczas wykonania, a nie w trakcie kompilacji. Jest to bezpośrednio związane z oddzieleniem logiki aplikacji od kodu. Zmniejsza 

\paragraph{Spring Security}

Spring Security

\section{Google Web Toolkit (ExtGWT)}
zalety gwt i alternatywy (inne gwt, UiBinder)
\section{Hibernate}
zalety hibernate i alternatywy
\section{System klas i uprawnień użytkowników}
zalety naszego i alternatywy (ldap)
\section{Środowisko pracy}
\subsection{Git}
Jest to rozproszony system kontroli wersji. Git jest opublikowany na licencji GNU GPL w wersji 2.
\subsection{Gerrit}
todo
\subsection{Jira}
Jira jest to oprogramowanie firmy Atlassian służące do zarządzania projektem oraz śledzenia błędów. 
Narzędzie to "łączy zespół z wykonywanym zadaniem".


\paragraph{Zarządzanie projektem} 

Podczas realizacji projektu wykorzystaliśmy metodykę agile'ową, która jest wspierana przez Jirę. 
Umożliwiło nam na łatwe i wygodne zarządzanie projektem. W szczególności korzystaliśmy z :
\begin{itemize}
\item Wersjonowania wydań - dzięki temu mogliśmy wyznaczać kolejne terminy realizacji danych wersji 
  dało nam to możliwość kontrolowania postępu prac i w razie potrzeby szybko reagować na zastoje;
\item Tworzenia raportów - informowało nas o stanie realizacji projektu oraz o postępie prac przez poszczególnych członków zespołu;
\item Tworzenia zadań - przydzielanie poszczególnym członkom zespołu, ustawianie terminów, komentowanie, oraz przydzielanie zasobów 
umożliwia wygodną komunikację oraz szybkie odnalezienie się w projekcie i w zadaniu jakie zostało nam przydzielone do realizacji;
\item Plansz agile'owych - np. Planning Board, Task Board w przejrzysty sposób informowały nas o stanie realizacji poszczególnych zadań oraz o 
etapie prac nad daną wersją projektu, umożliwiały prostą i wygodną edycję tych danych wspierając przyjętą przez nas metodykę agile'ową
\item Dashboardów - podobnie jak w podpunktach powyżej służyły do informowania nas o stanie realizacji projektu i o pracy poszczególnych członków zespołu
\end{itemize}

\paragraph{Zintegrowanie z innymi komponentami}
Jira jest integrowalna z innymi produktami Atlassiana, a w szczególności wykorzystywanym przez nas Confluencem.
Ponadto zintegrowaliśmy Jirę z kontem pocztowym (gmail) dzięki czemu otrzymywaliśmy powiadomienia o przydzielonych zadaniach, bądź innych zmianach w projekcie.


Bardzo istotna była integracja z systemem kontroli wersji oraz z webowym narzędziem do przeglądu kodu.
Jira umożliwia integrację z najbardziej popularnymi systemami jak : GIT, SVN, CVS, Clearcase i inne.
W naszym przypadku wykorzystaliśmy GITa, a do przeglądu kodu za pomocą przeglądarki używaliśmy Gerrita. 
Dzięki systemowi znaczników (przypisaniu numeru Zadania z Jiry do commita) możliwy był szybki podgląd zmian w Gerricie za pomocą automatycznie dodawanych linków do Zadań w Jirze.
Integrowało to zmiany w kodzie z postępem prac umieszczonym w schematach Jiry.  

\paragraph{Produkty konkurencyjne}
Jednym z produktów konkurencyjnych, których wykorzystanie rozważaliśmy jest stworzony na wydziale MIM Uniwersytetu Warszawskiego JLoXiM Redmine.
Produkt ten posiada większość z funkcjonalności jakie posiada Jira. W szczególności umożliwia tworzenie zadań, przydzielanie ich do członków zespołu, 
konfiguracją uprawnień użytkowników i zarządzanie całym projektem. Analogicznie jak Jira generuje raporty oraz w przejrzysty sposób pokazuje stan 
realizacji projektu.
Postanowiliśmy, jednakże wykorzystać produkt Atlassianowy, ponieważ część członków zespołu pracowała pod tymi narzędziami, Jira była łatwiej integrowalna
 z innymi produktami oraz posiadała bardziej przejrzysty i zaawansowany interfejs użytkownika.
\subsection{Confluence}
Confluence jest narzędziem dla zespołów służącym do uproszczenia współpracy, dzielenia się wiedzą, dyskusją pomysłów.
W naszym projekcie był on wykorzystywany w szczególności podczas etapu projektowania systemu oraz przy tworzeninu dokumentacji.
Confluence zapewnia wiele wygodnych rozwiązań.
\begin{itemize}
\item Przejrzysty podział na projekty - tworzenie dynamicznych stron , podstron , dowiązań, projektowanie dashboardów.
\item Proste w obsłudze i wygodne wiki - dodawanie linków, mockapów, filmów, plików.
\item Udostępnia pluginy do tworzenia schematów i diagramów np.: Gliffy
\item Integracja z Officem  
\end{itemize}
\subsection{Maven}
todo (to Wy wiecie)
\subsection{Eclipse}
todo (netbeans)
\chapter{Praca}
\section{Projektowanie systemu}
Wiele niespełnionych godzin, brak klienta z którym można pogadać.
\section{Przygotowanie środowiska pracy}
Problemy z serwerem, podział pracy i opłat pomiędzy dwa zespoły.
\section{Problemy z użytymi narzędziami}
\subsection{Spring + ExtGWT}
Problemy z integracją, kilka wieczorów, kilka problemów, ostatecznie wszystko działa sprawnie.
\subsection{Hibernate}
Dużo nieprzespanych nocy.
\subsection{Data Transfer Object}
Problem i jego rozwiązanie
% te dwa podpunkty można ując inaczej
\section{Zasoby czasowe i organizacja pracy}
Wszyscy mieszkamy blisko, wszyscy się znamy, łatwo się spotkać, łatwo dyskutować, a mimo to musieliśmy się spotykać w ferie i weekendy, ponieważ nigdy nie było czasu.
\section{Komunikacja w zespole}
Znamy swoje możliwości, jesteśmy blisko, a mimo to czasami nie dało się dogadać. Ciężko zwrócić komuś uwagę, ciężko coś wymóc, ze względu na osobiste relacje. Każdy inaczej reaguje na problemy (jedni się martwią, inni się denerwują, inni nie dadzą sobie nic powiedzieć), ale na szczęście po kilku godzinach znowu wszyscy się lubią. 
\section{Widoki na przyszłość}
Dużo umiemy, mamy możliwość przekształcenia aplikacji w cokolwiek, blebleble.

\chapter{Podział pracy w zespole}
todo

\chapter{Załączone pliki}
todo

\begin{comment}

\chapter*{Wprowadzenie}
\addcontentsline{toc}{chapter}{Wprowadzenie}

Blabalizator ró¿nicowy jest podstawowym narzêdziem blabalii
fetorycznej.  Dlatego naukowcy z~ca³ego ¶wiata prze¶cigaj± siê
w~próbach efektywnej implementacji.  Opracowana przez prof. Fifaka
teoria fetorów $\sigma$-$\rho$ otwiera w~tej dziedzinie nowe
mo¿liwo¶ci.  Wykorzystujemy je w~niniejszej pracy.

Przystêpne wprowadzenie do blabalii fetorycznej mo¿na znale¼æ w~pracy
Fifaka i~Gryzogrzechotalskiego \cite{ffgg}.  Dlatego w~niniejszym
tek¶cie ograniczymy siê do przypomnienia pojêæ podstawowych.

Praca sk³ada siê z~piêciu rozdzia³ów i~dodatków.
W~rozdziale~\ref{r:pojecia} przypomniano podstawowe pojêcia blabalii
fetorycznej.  Dotychczasowe próby implementacji blablizatora
ró¿nicowego zestawiono w~rozdziale~\ref{r:losers}.
Rozdzia³~\ref{r:fifak} przedstawia teoriê Fifaka i~wyja¶nia sposób jej
wykorzystania w~implementacji blabalizatora.  W~rozdziale \ref{r:impl}
przedstawiono algorytm blabalizy i~realizuj±cy go program komputerowy.
Ostatni rozdzia³ zawiera przemy¶lenia dotycz±ce mo¿liwego wp³ywu
dostêpno¶ci efektywnej blabalizy numerycznej na rozwój blabalii
fetorycznej.  W~dodatkach umieszczono najciekawszy fragment programu,
przyk³adowe dane i~wyniki dzia³ania programu.

\chapter{Podstawowe pojêcia}\label{r:pojecia}

Pojêciem pierwotnym blabalii fetorycznej jest \emph{blaba}.
Blabali¶ci nie podaj± jego definicji, mówi±c za Ciach-Pfe t-\=am
K\^un (fooistyczny mêdrzec, XIX w. p.n.e.):
\begin{quote}
  Blaba, który jest blaba, nie jest prawdziwym blaba.

\raggedleft\slshape t³um. z~chiñskiego Robert Blarbarucki
\end{quote}

\section{Definicje}

Oto dwie definicje wprowadzaj±ce podstawowe pojêcia blabalii
fetorycznej:

\begin{defi}\label{skupienie}
  Silny, zwarty i gotowy fetor bazowy nazwiemy \emph{skupieniem}.
\end{defi}

\begin{defi}\label{fetor}
  \emph{Fetorem} nazwiemy skupienie blaba spe³niaj±ce nastêpuj±cy
  \emph{aksjomat reperkusatywno¶ci}:
  $$\forall \mathcal{X}\in Z(t)\ \exists
  \pi\subseteq\oint_{\Omega^2}\kappa\leftrightarrow 42$$
\end{defi}


\section{Blabalizator ró¿nicowy}

Teoretycy blabalii (zob. np. pracê~\cite{grglo}) zadowalaj± siê
niekonstruktywnym opisem natury fetorów.

Podstawowym narzêdziem blabalii empirycznej jest blabalizator
ró¿nicowy.  Przyrz±d ten pozwala w~sposób przybli¿ony uzyskaæ
wspó³czynniki rozk³adu G³ombaskiego dla fetorów bazowych
i~harmonicznych.  Praktyczne znaczenie tego procesu jest oczywiste:
korzystaj±c z~reperkusatywno¶ci pozwala on przej¶æ do przestrzeni
$\Lambda^{\nabla}$, a~tym samym znale¼æ retroizotonalne wspó³czynniki
semi-quasi-celibatu dla klatek Rozkoszy (zob.~\cite{JR}).

Klasyczne algorytmy dla blabalizatora ró¿nicowego wykorzystuj±:
\begin{enumerate}
\item dualizm falowo\dywiz korpuskularny, a w szczególno¶ci
  \begin{enumerate}
  \item korpuskularn± naturê fetorów,
  \item falow± naturê blaba,
  \item falowo\dywiz korpuskularn± naturê gryzmo³ów;
  \end{enumerate}
\item postêpuj±c± gryzmolizacjê poszczególnych dziedzin nauki, w
  szczególno¶ci badañ systemowych i rozcieñczonych;
\item dynamizm fazowy stetryczenia parajonizacyjnego;
\item wreszcie tradycyjne opozycje:
  \begin{itemize}
  \item duch --- bakteria,
  \item mieæ --- chcieæ,
  \item my¶l --- owsianka,
  \item parafina --- durszlak\footnote{Wiêcej o tym przypadku --- patrz
      prace Gryzybór\dywiz G³ombaskiego i innych teoretyków nurtu
      teoretyczno\dywiz praktycznego badañ w~Instytucie Podstawowych
      Problemów Blabalii w~Fifie.},
  \item logos --- termos%\footnote{Szpotañski}
  \end{itemize}
  z w³a¶ciwym im przedziwym dynamizmem.
\end{enumerate}

\begin{figure}[tp]
  \centering
  \framebox{\vbox to 4cm{\vfil\hbox to
      7cm{\hfil\tiny.\hfil}\vfil}}
  \caption{Artystyczna wizja blaba w~obrazie wêgierskiego artysty
    Josipa~A. Rozkoszy pt.~,,Blaba''}
\end{figure}

\chapter{Wcze¶niejsze implementacje blabalizatora
  ró¿nicowego}\label{r:losers}

\section{Podej¶cie wprost}

Najprostszym sposobem wykonania blabalizy jest si³owe przeszukanie
ca³ej przestrzeni rozwi±zañ.  Jednak, jak ³atwo wyliczyæ, rozmiar
przestrzeni rozwi±zañ ro¶nie wyk³adniczo z~liczb± fetorów bazowych.
Tak wiêc przegl±d wszystkich rozwi±zañ sprawdza siê jedynie dla bardzo
prostych przestrzeni lamblialnych.  Oznacza to, ¿e taka metoda ma
niewielkie znaczenie praktyczne --- w~typowym przypadku z~¿ycia trzeba
rozwa¿aæ przestrzenie lamblialne wymiaru rzêdu 1000.

W~literaturze mo¿na znale¼æ kilka prób opracowania heurystyk dla
problemu blabalizy (por. \cite{heu}).  Korzystaj±c z~heurystyk daje
siê z~pewnym trudem dokonaæ blabalizy w~przestrzeni o~np.~500 fetorach
bazowych.  Nale¿y jednak pamiêtaæ, ¿e heurystyka nie daje gwarancji
znalezienia najlepszego rozwi±zania.  Fifak w~pracy~\cite{ff-sr}
podaje, jak dla dowolnie zadanej funkcji oceniaj±cej skonstruowaæ
dane, dla których rozwi±zanie wygenerowane przez algorytm heurystyczny
jest dowolnie odleg³e od rzeczywistego.

\section{Metody wykorzystuj±ce teoriê G³ombaskiego}

Teoria G³ombaskiego (zob.~\cite{grglo}) dostarcza eleganckiego
narzêdzia opisu przej¶cia do przestrzeni $\Lambda^{\nabla}$.
Wystarczy mianowicie przedstawiæ fetory bazowe wyj¶ciowej przestrzeni
lamblialnej w~nieskoñczonej bazie tak zwanych wy¿szych aromatów.
(Formaln± definicjê tego pojêcia przedstawiê w~rozdziale po¶wiêconym
teorii Fifaka).  Podstawow± cech± wy¿szych aromatów jest ulotno¶æ.  To
za¶ oznacza, ¿e odpowiednio dobieraj±c wspó³czynniki przej¶cia do
przestrzeni wy¿szych aromatów mo¿na zagwarantowaæ dowoln± z~góry
zadan± dok³adno¶æ przybli¿onego rozwi±zania problemu blabalizy.

Oczywi¶cie ze wzglêdu na nieskoñczony wymiar przestrzeni wy¿szych
aromatów koszt poszukiwania wspó³czynników blabalizy jest liniowy ze
wzglêdu na wymiar wyj¶ciowej przestrzeni lamblialnej.

\section{Metody wykorzystuj±ce w³asno¶ci fetorów $\sigma$}

Najchêtniej wykorzystywan± przestrzeni± wy¿szych aromatów jest
przestrzeñ fetorów~$\sigma$.  Fetory $\sigma$ daj± szczególnie prost±
bazê podprzestrzeni wid³owej.  Wi±¿e siê to z~faktem, ¿e w~tym przypadku
fetory harmoniczne wy¿szych rzêdów s± pomijalne (rzêdu $2^{-t^3}$,
gdzie $t$ jest wymiarem wyj¶ciowej przestrzeni lamblialnej).

Niestety z~fetorami $\sigma$ wi±¿e siê te¿ przykre ograniczenie: mo¿na
wykazaæ (zob.~\cite[s. 374]{ff-sr}), ¿e dla dowolnie dobranej bazy
w~podprzestrzeni wid³owej istnieje ograniczenie dolne w~metryce sierpa
na odleg³o¶æ rzutu dok³adnego rozwi±zania problemu blabalizy na
podprzestrzeñ wid³ow±.  Poniewa¿ rzut ten stanowi najlepsze
przybli¿one rozwi±zanie, jakie mo¿na osi±gn±æ nie naruszaj±c aksjomatu
reperkusatywno¶ci, wiêc istnieje pewien nieprzekraczalny próg
dok³adno¶ci dla blabalizy wykonanej przez przej¶cie do przestrzeni
fetorów $\sigma$.  Warto¶æ retroinicjaln± tego progu nazywa siê
\textit{reziduum blabicznym}.

\chapter{Teoria fetorów $\sigma$-$\rho$}\label{r:fifak}

G³ównym odkryciem Fifaka jest, ¿e fetor suprakowariantny mo¿e
gryzmolizowaæ dowolny idea³ w~podprzestrzeni wid³owej przestrzeni
lamblialnej funkcji Rozkoszy.

Udowodnienie tego faktu wymaga³o wykorzystania twierdzeñ pochodz±cych
z~kilku niezale¿nych teorii matematycznych (zob. na przyk³ad:
\cite{russell,spyrpt,JR,beaman,hopp,srinis}).  Jednym z~filarów
dowodu jest teoria odwzorowañ owalnych Leukocyta (zob.~\cite{leuk}).

Znaczenie twierdzenia Fifaka dla problemu blabalizy polega na tym, ¿e
znaj±c retroizotonalne wspó³czynniki dla klatek Rozkoszy mo¿na
przeprowadziæ fetory bazowe na dwie nieskoñczone bazy fetorów $\sigma$
w~przestrzeni $K_7$ i~fetorów $\rho$ w~odpowiedniej
quasi-quasi-przestrzeni równoleg³ej (zob.~\cite{hopp}).  Zasadnicza
ró¿nica w~stosunku do innych metod blabalizy polega na tym, ¿e
przedstawienie to jest dok³adne.

\chapter{Dokumentacja u¿ytkowa i~opis implementacji}\label{r:impl}

Program przygotowany dla systemu operacyjnego M\$ Windows uruchamia
siê energicznym dwumlaskiem na jego ikonce w~folderze
\verb+\\FIDO\FOO\BLABA+.  Nastêpnie kolistym ruchem rêki nale¿y
naprowadziæ kursor na menu \texttt{Blabaliza} i~uaktywniæ pozycjê
\texttt{Otwórz plik}.  Po wybraniu pliku i~zatwierdzeniu wyboru
przyciskiem \texttt{OK} rozpocznie siê proces blabalizy.  Wyniki
zostan± zapisane w~pliku o~nazwie \texttt{99-1a.tx.43} w~bie¿±cym
folderze.

\chapter{Podsumowanie}

W~pracy przedstawiono pierwsz± efektywn± implementacjê blabalizatora
ró¿nicowego.  Umiejêtno¶æ wykonania blabalizy numerycznej dla danych
,,z ¿ycia'' stanowi dla blabalii fetorycznej podobny prze³om, jak dla
innych dziedzin wiedzy stanowi³o og³oszenie teorii Miko³aja Kopernika
i~Gryzybór G³ombaskiego.  Z~pewno¶ci± w~rozpocznynaj±cym siê XXI wieku
bêdziemy obserwowaæ rozkwit blabalii fetorycznej.

Trudno przewidzieæ wszystkie nowe mo¿liwo¶ci, ale te co bardziej
oczywiste mo¿na wskazaæ ju¿ teraz.  S± to:
\begin{itemize}
\item degryzmolizacja wieñców telecentrycznych,
\item realizacja zimnej reakcji lambliarnej,
\item loty celulityczne,
\item dok³adne obliczenie wieku Wszech¶wiata.
\end{itemize}

\section{Perspektywy wykorzystania w~przemy¶le}

Ze wzglêdu na znaczenie strategiczne wyników pracy ten punkt uleg³
utajnieniu.

\appendix

\chapter{G³ówna pêtla programu zapisana w~jêzyku T\=oFoo}

\begin{verbatim}
[[foo]{,}[[a3,(([(,),{[[]]}]),
  [1; [{,13},[[[11],11],231]]].
  [13;[!xz]].
  [42;[{,x},[[2],{'a'},14]]].
  [br;[XQ*10]].
 ), 2q, for, [1,]2, [..].[7]{x}],[(((,[[1{{123,},},;.112]],
        else 42;
   . 'b'.. '9', [[13141],{13414}], 11),
 [1; [[134,sigma],22]].
 [2; [[rho,-],11]].
 )[14].
 ), {1234}],]. [map [cc], 1, 22]. [rho x 1]. {22; [22]},
       dd.
 [11; sigma].
        ss.4.c.q.42.b.ll.ls.chmod.aux.rm.foo;
 [112.34; rho];
        001110101010101010101010101010101111101001@
 [22%f4].
 cq. rep. else 7;
 ]. hlt
\end{verbatim}

\chapter{Przyk³adowe dane wej¶ciowe algorytmu}

\begin{center}
  \begin{tabular}{rrr}
    $\alpha$ & $\beta$ & $\gamma_7$ \\
    901384 & 13784 & 1341\\
    68746546 & 13498& 09165\\
    918324719& 1789 & 1310 \\
    9089 & 91032874& 1873 \\
    1 & 9187 & 19032874193 \\
    90143 & 01938 & 0193284 \\
    309132 & $-1349$ & $-149089088$ \\
    0202122 & 1234132 & 918324098 \\
    11234 & $-109234$ & 1934 \\
  \end{tabular}
\end{center}

\chapter{Przyk³adowe wyniki blabalizy
    (ze~wspó³czynnikami~$\sigma$-$\rho$)}

\begin{center}
  \begin{tabular}{lrrrr}
    & Wspó³czynniki \\
    & G³ombaskiego & $\rho$ & $\sigma$ & $\sigma$-$\rho$\\
    $\gamma_{0}$ & 1,331 & 2,01 & 13,42 & 0,01 \\
    $\gamma_{1}$ & 1,331 & 113,01 & 13,42 & 0,01 \\
    $\gamma_{2}$ & 1,332 & 0,01 & 13,42 & 0,01 \\
    $\gamma_{3}$ & 1,331 & 51,01 & 13,42 & 0,01 \\
    $\gamma_{4}$ & 1,332 & 3165,01 & 13,42 & 0,01 \\
    $\gamma_{5}$ & 1,331 & 1,01 & 13,42 & 0,01 \\
    $\gamma_{6}$ & 1,330 & 0,01 & 13,42 & 0,01 \\
    $\gamma_{7}$ & 1,331 & 16435,01 & 13,42 & 0,01 \\
    $\gamma_{8}$ & 1,332 & 865336,01 & 13,42 & 0,01 \\
    $\gamma_{9}$ & 1,331 & 34,01 & 13,42 & 0,01 \\
    $\gamma_{10}$ & 1,332 & 7891432,01 & 13,42 & 0,01 \\
    $\gamma_{11}$ & 1,331 & 8913,01 & 13,42 & 0,01 \\
    $\gamma_{12}$ & 1,331 & 13,01 & 13,42 & 0,01 \\
    $\gamma_{13}$ & 1,334 & 789,01 & 13,42 & 0,01 \\
    $\gamma_{14}$ & 1,331 & 4897453,01 & 13,42 & 0,01 \\
    $\gamma_{15}$ & 1,329 & 783591,01 & 13,42 & 0,01 \\
  \end{tabular}
\end{center}

\begin{thebibliography}{99}
\addcontentsline{toc}{chapter}{Bibliografia}

\bibitem[Bea65]{beaman} Juliusz Beaman, \textit{Morbidity of the Jolly
    function}, Mathematica Absurdica, 117 (1965) 338--9.

\bibitem[Blar16]{eb1} Elizjusz Blarbarucki, \textit{O pewnych
    aspektach pewnych aspektów}, Astrolog Polski, Zeszyt 16, Warszawa
  1916.

\bibitem[Fif00]{ffgg} Filigran Fifak, Gizbert Gryzogrzechotalski,
  \textit{O blabalii fetorycznej}, Materia³y Konferencji Euroblabal
  2000.

\bibitem[Fif01]{ff-sr} Filigran Fifak, \textit{O fetorach
    $\sigma$-$\rho$}, Acta Fetorica, 2001.

\bibitem[G³omb04]{grglo} Gryzybór G³ombaski, \textit{Parazytonikacja
    blabiczna fetorów --- nowa teoria wszystkiego}, Warszawa 1904.

\bibitem[Hopp96]{hopp} Claude Hopper, \textit{On some $\Pi$-hedral
    surfaces in quasi-quasi space}, Omnius University Press, 1996.

\bibitem[Leuk00]{leuk} Lechoslav Leukocyt, \textit{Oval mappings ab ovo},
  Materia³y Bia³ostockiej Konferencji Hodowców Drobiu, 2000.

\bibitem[Rozk93]{JR} Josip A.~Rozkosza, \textit{O pewnych w³asno¶ciach
    pewnych funkcji}, Pó³nocnopomorski Dziennik Matematyczny 63491
  (1993).

\bibitem[Spy59]{spyrpt} Mrowclaw Spyrpt, \textit{A matrix is a matrix
    is a matrix}, Mat. Zburp., 91 (1959) 28--35.

\bibitem[Sri64]{srinis} Rajagopalachari Sriniswamiramanathan,
  \textit{Some expansions on the Flausgloten Theorem on locally
    congested lutches}, J. Math.  Soc., North Bombay, 13 (1964) 72--6.

\bibitem[Whi25]{russell} Alfred N. Whitehead, Bertrand Russell,
  \textit{Principia Mathematica}, Cambridge University Press, 1925.

\bibitem[Zen69]{heu} Zenon Zenon, \textit{U¿yteczne heurystyki
    w~blabalizie}, M³ody Technik, nr~11, 1969.

\end{thebibliography}

\end{comment}

\end{document}


%%% Local Variables:
%%% mode: latex
%%% TeX-master: t
%%% coding: latin-2
%%% End:
